
\documentclass[12pt,a4paper,onecolumn]{article}
\usepackage[ngerman]{babel}
\usepackage[utf8]{inputenc}
\usepackage{datetime}
\usepackage{tabularx}
% \usepackage[T1]{fontenc}
% \usepackage[sc]{mathpazo}
\usepackage[autostyle]{csquotes}
% \usepackage{graphicx}
% \usepackage{scrhack} % necessary for listings package
% \usepackage{listings}
% \usepackage{lstautogobble}
% \usepackage{tikz}
% \usepackage{booktabs}
% \usepackage[final]{microtype}
\usepackage{caption}
\usepackage{hyperref} % hidelinks removes colored boxes around references and links
% \usepackage{comment}

\usepackage{enumitem}


\newcommand\titleofdoc{StayHealthy} % Put your document title in this argument
\newcommand\GroupName{Team 6} % Put your group name here. If you are the only member of the group, just put your name

\begin{document}
\begin{titlepage}
   \begin{center}
        \vspace*{4cm} % Adjust spacings to ensure the title page is generally filled with text

        \Huge{\titleofdoc} 

        \vspace{0.5cm}
        \LARGE{Projektplan}
            
        \vspace{3 cm}
        \Large{\GroupName}
       
        \vspace{0.25cm}
        \large{Khader AlHamed, Marco Klein\\Denis Manherz, Andreas Wirth}
       
        \vspace{3 cm}
        \Large{\today}% change date format to dd.mm.yyyy
        
        \vspace{0.25 cm}
        \Large{Software Praktikum}
       

       \vfill
    \end{center}
\end{titlepage}
\setcounter{page}{2}
\tableofcontents
\newpage

\section{Dokumentinformationen} 
\subsection{Änderungsgeschichte}
\begin{center}
\begin{tabular}{ |c|c|c|c| } 
 \hline
 Datum & Version & Änderung & Autor\\ 
 \hline
 25.03.2022 & 0.0 & Erstellung & Manherz \\ 
 \hline
 27.03.2022 & 1.0 & Grobe Ausarbeitung & Manherz\\
 \hline
 31.03.2022 & 1.1 & Ergänzungen allgemein & Wirth\\
 \hline
 02.04.2022 & 1.2 & Ergänzung Arbeitspakete & Wirth\\
 \hline
 04.04.2022 & 1.3 & Anpassungen nach Rücksprache & Wirth\\
 \hline
 26.04.2022 & 1.4 & Meilensteine überarbeitet & Wirth\\
 \hline
 28.04.2022 & 1.5 & Ergänzung unter 3.2, 6 und 9 & Wirth\\
 \hline
 29.04.2022 & 1.6 & Anpassungen unter 5.2 und 7 & Wirth\\
 \hline
\end{tabular}
\end{center}

\section{Einführung}
\subsection{Definitionen und Abkürzungen}
\textbf{Ernährungsverhalten} - Ernährungsbezogene Handlungen die Menschen im Alltag vollziehen.\\
\textbf{Gesundheitsverhalten} - Handlungen von gesunden Menschen die das Risiko von Erkrankungen nachweislich senken oder welche die Gesundheit positiv beeinflussen.
\subsection{Referenzen}
\href{https://sceweb.uhcl.edu/helm/RationalUnifiedProcess/webtmpl/templates/mgmnt/rup_sdpln_sp.htm}{sceweb.uhcl.edu - Software Development Plan}
\subsection{Übersicht}
Der Inhalt dieses Dokuments beschreibt den Umfang des Projekts StayHealthy, sowie die zeitlichen Rahmenbedingungen und die generelle Organisation. 
\section{Projektübersicht}
Mit Hilfe der StayHealthy Software kann der Benutzer Trainings- und Ernährungspläne erstellen. In ihrem Zeitplan können sie diese dann ansehen bzw. durchführen. Im allgemeinen soll dies dem Benutzer dabei helfen sich an eine Ernährungs- bzw. Trainingsroutine zu halten. 
\subsection{Zweck und Ziel}
StayHealthy soll dazu dienen das Gesundheitsverhalten des Benutzers zu verbessern indem es sein Trainings- und Ernährungsverhalten überwacht und ihm sportliche Aktivitäten vorschlägt.
\subsection{Annahmen und Einschränkungen}
Eine funktionsfähige Datenbankanbindung ist für dieses Projekt unabdinglich, um den Fortschritt des Benutzer über die Zeit nachvollziehen zu können und um entsprechende Vorschläge anzubieten.\\
Der Projektverlauf wird stetig in den wöchentlichen Meetings überprüft (Vergleich Soll - Ist) und wird ggf. angepasst.\\
Jedes Teammitglied ist für die individuelle Zeiterfassung seiner aufgebrachten Arbeitszeit selbst zuständig. Diese wird jedoch in einem gemeinsamen Dokument gespeichert, um den nötigen Überblick zu behalten.
\subsection{Arbeitsergebnisse}
\begin{tabularx}{\textwidth}{ |l|X| } 
 \hline
\textbf{Arbeitsergebnis}  & \textbf{Beschreibung}\\ 
 \hline
 StayHealthy & Die fertige Software\\ 
 \hline
 StayHealthy-Code & Kommentierter Code\\
 \hline
 Projektdokumentation & Beschreibt wie die Software entwickelt werden soll\\
 \hline
 Softwaredokumentation & Erklärt die Funktionen der Software sowie ihre Anwendung\\
 \hline
 UML-Diagramme & Alle Diagramme die für das Verständnis der Architektur notwendig sind \\ 
 \hline
 Zeitpool & Die vollständige Zeiterfassung\\
 \hline
 Testdokumente & Strukturierte Tests der Software\\
 \hline
\end{tabularx}

\section{Projektorganisation}

\subsection{Organisationsstruktur}
Teamübersicht:
\vspace{0.2cm}\\
\begin{tabularx}{\textwidth}{l X}
Khader Alhamed & khader1.alhamed@st.oth-regensburg.de\\
Marco Klein & marco.klein@st.oth-regensburg.de\\
Denis Manherz & denis.manherz@st.oth-regensburg.de\\
Andreas Wirth & andreas2.wirth@st.oth-regensburg.de\\
\end{tabularx}
\vspace{0.2cm}\\
\begin{tabular}{|l|l|}
\hline
     \textbf{Verantwortungsbereich} & \textbf{Verantwortlicher}\\
     \hline
     Projekt Management & Klein\\
     \hline
     Systemadministration &  Manherz\\
     \hline 
     Software Architektur & AlHamed \\
     \hline
     Qualitätsmanagement & Wirth\\
     \hline
     User Interface Design & Alle\\
     \hline
     Implementierung & Alle\\
     \hline
     Code Review & Alle\\
     \hline
\end{tabular}
\subsection{Externe Schnittstellen}
Stetiger Ansprechpartner und Betreuer des Projekts ist Prof. Dr. Axel Doering. Als Team nehmen wir bei Prof. Dr. Doering Beratungstermine war und besprechen Projektergebnisse in Reviewterminen. 

\section{Management Abläufe}
\subsection{Projekt Kostenvoranschlag}
Für das Projekt entstehen vorerst keine Kosten, da es lokal auf einem System installiert werden kann. Falls die Datenbank später ausgelagert wird benötigen wir
einen Backendserver. Dieser steht über die virtuelle Maschine, auf die wir im Rahmen des Projekts Zugriff haben, zur Verfügung. Nach Abschluss des Projekts können Kosten für das Betreiben dieses Servers anfallen.
\subsection{Projektplan}
Die zeitliche Dauer des Projekts umfasst 16 Wochen, wobei die Schlussabgabe am 07.07.2022 ist. Ein genauer Termin der Abschlusspräsentation ist zur Zeit noch nicht bekannt. Die genaue Zeiteinteilung der einzelnen Phasen und Iterationen werden im Zeitplan aufgeführt
\subsubsection{Zeitplan}
Der angestrebte Aufwand pro Teammitglied ist 150 Stunden, folglich eine gesamte Arbeitszeit von 600 Stunden. Daraus ergibt sich im Durschnitt eine wöchentliche Arbeitszeit von 9-10 Stunden für jeden Teilnehmer. In allen Arbeitspaketen ist Fehlermanagement mit inbegriffen. Die Angaben im Zeitplan beziehen sich auf Stunden.\\

\subsubsection{Arbeitspakete Allgemeine Aufwendungen}
\begin{tabular}{|l|c|c|c|c|c|c|}
\hline
      \textbf{\#} & \textbf{Arbeitspaket} & \textbf{DM} & \textbf{MK} & \textbf{AW} & \textbf{KA} & \textbf{Gesamt} \\

\hline
01 & Visiondokument & & 5 & & & 5\\
\hline
02 & Anforderungsspezifikation & 10 & & & & 10\\
\hline
03 & Projektplan & 3 & & 3 & 3 & 9\\
\hline
04 & Use Cases & & 8 & & 8 & 16 \\
\hline
05 & UML-Diagramme & & 8 & 4 & 4 & 16 \\
\hline
06 & Installation und Konfiguration & 2 & 2 & 2 & 2 & 8\\
\hline
07 & Einarbeitung & 5 & 5 & 5 & 5 & 20\\
\hline
08 & Datenbank aufsetzen & 4 & & 4 & & 8\\
\hline
09 & Softwaredokumentation & & 8 & & 8 & 16\\
\hline
10 & Code Review & 10 & 10 & 10 & 10 & 40\\
\hline
11 & Tests & 15 & & 15 & & 30\\
\hline
12 & Meetings & 32 & 32 & 32 & 32& 128\\
\hline
13 & Abschlusspräsentation & 2 & 2 & 2 & 2 & 8\\
\hline
\hline
 & Gesamt & 83 & 80 & 77 & 74 & 314\\
\hline
\end{tabular}

\subsubsection{Arbeitspakete Use-Cases}
\begin{tabular}{|l|c|c|c|c|c|c|}
\hline
      \textbf{\#} & \textbf{Arbeitspaket} & \textbf{DM} & \textbf{MK} & \textbf{AW} & \textbf{KA} & \textbf{Gesamt} \\



\hline
14 & Ansichten erstellen & & 10 & & 10 & 20\\
\hline
15 & Benutzer registrieren & 4 & & 4 & & 8\\
\hline
16 & Benutzer anmelden & 4 & & 4 & & 8\\
\hline
17 & Profil bearbeiten & 4 & & 4 & & 8\\
\hline
18 & Profil aufwerten & 4 & & 4 & & 8\\
\hline
19 & Grundumsatz berechnen &  & 2 & & 2 & 4\\
\hline
20 & Mahlzeit eingeben & & 6 & & 6 & 12\\
\hline
21 & Kalorien berechnen & & 4 & & 4 & 8\\
\hline
22 & UC Trainingsplan erstellen & & 10 & & 10 & 20\\
\hline
23 & Auto Trainingsplan erstellen & 10 & & 10 & & 20\\
\hline
24 & Trainingseinheit erstellen & 10 &  & 10 &  & 20\\
\hline
25 & Ernährungsplan erstellen & & 10 & & 10 & 20 \\
\hline
26 & Statistik erstellen & 10 & & 10 & & 20\\
\hline
27 & Erinnerung senden & 4 & & 4 & & 8\\
\hline
28 & Termine bearbeiten & 4 & & 4 & & 8\\
\hline
29 & Training durchführen & & 8 & & 8 & 16\\
\hline
\hline
 & Gesamt & 54 & 50 & 54 & 50 & 208\\
\hline
\end{tabular}

\subsubsection{Zeitaufwand Gesamt}

\begin{tabular}{|l|c|c|c|c|c|c|}
\hline
      \textbf{\#} & \textbf{Arbeitspaket} & \textbf{DM} & \textbf{MK} & \textbf{AW} & \textbf{KA} & \textbf{Gesamt} \\
\hline
30 & Puffer & 25 & 25 & 25 & 25 & 100\\
\hline
 & Allgemeine Aufwendungen & 83 & 80 & 77 & 74 & 314\\
\hline
 & Use-Cases & 54 & 50 & 54 & 50 & 208\\


\hline
\hline
 & Gesamt & 162 & 155 & 156 & 149 & 622\\
\hline

\end{tabular}
\subsubsection{Iterationsplanung/Meilensteine}
Die Entwicklung erfolgt interativ. Angestrebte Meilensteine werden ständig überprüft und wenn nötig agil angepasst.
Einzelne Code Module werden stetig dokumentiert und getestet.\\
Deadlines orientieren sich nach den vorgebenen Abgabeterminen/Meilensteinen von Herrn Doering.\\

\begin{itemize}
    \item Meilenstein 00 ``Projektantrag'' 19.03.2022
    \begin{itemize}
        \item[-] Projektantrag wurde eingereicht
    \end{itemize}
    
    \item Meilenstein 01 ``Projektplan'' 04.04.2022
    \begin{itemize}
        \item[-] Ausgearbeiteter Projektplan liegt vor
    \end{itemize}
  
    \item Meilenstein 02 ``Requirements'' 22.04.2022
    \begin{itemize}
        \item[-] Alle nötigen UML-Diagramme wurden erstellt
        \item[-] Datenbank wurde aufgesetzt, ist funktionsfähig und CRUD-Funktionalitäten sind implementiert
        \item[-] GUI wurde mit nötigen (noch nicht funktionsfähigten) Elementen (Buttons usw.) erstellt
    \end{itemize}
  
    \item Meilenstein 03 ``Architektur'' 19.05.2022
    \begin{itemize}
        \item[-] Benutzer kann Profil erstellen/bearbeiten/löschen
        \item[-] Profil kann aufgewertet werden
        \item[-] Mahlzeiten können eingegeben werden
        \item[-] Trainingsplan erstellen
        \item[-] Testplan für Code wurde erstellt
    \end{itemize}
    
    \item Meilenstein 04 ``Design'' 10.06.2022
    \begin{itemize}
        \item[-] Training kann vorgeschlagen werden.
        \item[-] Benutzer kann ein vorgeschlagenes Training durchführen.
        \item[-] Benutzer kann Ernährungs-/Trainingsplan erstellen
        \item[-] Benutzer kann Statistik anzeigen
        \item[-] Benutzer kann Zeitplan bearbeiten
    \end{itemize}
    
    \item Meilenstein 05 ``Schlussabgabe'' 07.07.2022
    \begin{itemize}
        \item[-] Finale Projektversion ist fertiggestellt
        \item[-] Software Dokumentation ist vorhanden
    \end{itemize}
    
    \item Meilenstein 06 ``Präsentation'' 19.07.2022
    \begin{itemize}
        \item[-] Abschlusspräsentation ist vorbereitet
    \end{itemize}
  
  
  
\end{itemize}


\subsubsection{Besprechungen}
Das gesamte Team trifft sich regelmäßig zu Teammeetings und zu den verpflichtenden Reviewterminen an der OTH. Bei Notwendigkeit können außerplanmäßige Teammeetings auch über eine Videokonferenz stattfinden. Die Sitzungen belaufen sich auf ca. zwei Stunden wöchentlich. Anstehende Aufgaben, bis zum nächsten Teammeeting werden verteilt, Unklarheiten sowie Fragen werden besprochen und gelöst. 
\subsubsection{Abgabe}
\begin{tabularx}{\textwidth}{|X|X|X|}
\hline
\textbf{Art} & \textbf{Abgabe spätestens} & \textbf{Inhalt}\\
\hline
Prototyp & 19.05.22 & siehe Meilenstein 03\\
\hline
\shortstack{Vorläufiges\\ Endprodukt} & 24.06.22 & siehe Meilenstein 04\\
\hline
\end{tabularx}

\section{Risiko Management}
Mögliche Risiken und Problemlösungen\\\\
\begin{tabularx}{\textwidth}{|X|X|}
\hline
    \textbf{Risiko} & \textbf{Lösungsansatz}\\
    \hline
    Datenbankanbindung nicht funktionsfähig & Datenbankanbindung rechtzeitig und ausgiebig testen, bei nicht-Funktionalität alternative DB integrieren\\
    \hline
    Unerwarteter (temporärer) Ausfall von Teammitglied (z.B. durch Krankheit) & Aufgaben an andere Mitglieder verteilen und ggf. Projektumfang anpassen\\
    \hline
\end{tabularx}

\section{Arbeitspakete}
Die Arbeitspakete werden im Laufe des Projekts angepasst.

\subsection{Arbeitspaket 01 Visiondokument}
\begin{tabularx}{\textwidth}{|l|X|}
\hline
     \textbf{Zeit} &  5 \\
     \hline
     \textbf{Ressource} & Marco Klein\\
     \hline
     \textbf{Inhalt} & Visiondokument \\
     \hline
     \textbf{Abhängigkeit} &  keine\\
\hline
\end{tabularx}

\subsection{Arbeitspaket 02 Anforderungsspezifikation}
\begin{tabularx}{\textwidth}{|l|X|}
\hline
     \textbf{Zeit} &  10\\
     \hline
     \textbf{Ressource} & Manherz\\
     \hline
     \textbf{Inhalt} & Anforderungsspezifikation \\
     \hline
     \textbf{Abhängigkeit} & Use Cases 7.4\\
\hline
\end{tabularx}

\subsection{Arbeitspaket 03 Projektplan}
\begin{tabularx}{\textwidth}{|l|X|}
\hline
     \textbf{Zeit} &  9\\
     \hline
     \textbf{Ressource} & AlHamed, Manherz, Wirth\\
     \hline
     \textbf{Inhalt} &  Ausgearbeiteter Projektplan\\
     \hline
     \textbf{Abhängigkeit} & keine\\
\hline
\end{tabularx}

\subsection{Arbeitspaket 04 Use Cases}
\begin{tabularx}{\textwidth}{|l|X|}
\hline
     \textbf{Zeit} &  16\\
     \hline
     \textbf{Ressource} & Klein, AlHamed\\
     \hline
     \textbf{Inhalt} &  Ausarbeitung der Use Cases\\
     \hline
     \textbf{Abhängigkeit} & keine\\
\hline
\end{tabularx}

\subsection{Arbeitspaket 05 UML-Diagramme}
\begin{tabularx}{\textwidth}{|l|X|}
\hline
     \textbf{Zeit} &  16\\
     \hline
     \textbf{Ressource} & AlHamed, Klein, Wirth\\
     \hline
     \textbf{Inhalt} &  Use Case Diagramm, Entity Relationship Modell\\
     \hline
     \textbf{Abhängigkeit} & keine \\
\hline
\end{tabularx}

\subsection{Arbeitspaket 06 Installation und Konfiguration}
\begin{tabularx}{\textwidth}{|l|X|}
\hline
     \textbf{Zeit} &  8\\
     \hline
     \textbf{Ressource} & Alle\\
     \hline
     \textbf{Inhalt} &  Installieren und Konfigurieren von benötigten Tools\\
     \hline
     \textbf{Abhängigkeit} & keine\\
\hline
\end{tabularx}

\subsection{Arbeitspaket 07 Einarbeitung}
\begin{tabularx}{\textwidth}{|l|X|}
\hline
     \textbf{Zeit} & 20 \\
     \hline
     \textbf{Ressource} & Alle\\
     \hline
     \textbf{Inhalt} & Einarbeitung in Benötigte Tools und Technologien \\
     \hline
     \textbf{Abhängigkeit} & keine \\
\hline
\end{tabularx}

\subsection{Arbeitspaket 08 Datenbank aufsetzen}
\begin{tabularx}{\textwidth}{|l|X|}
\hline
     \textbf{Zeit} & 8 \\
     \hline
     \textbf{Ressource} & Manherz, Wirth\\
     \hline
     \textbf{Inhalt} &  Erstellen von Tabellen\\
     \hline
     \textbf{Abhängigkeit} & keine\\
\hline
\end{tabularx}

\subsection{Arbeitspaket 09 Softwaredokumentation}
\begin{tabularx}{\textwidth}{|l|X|}
\hline
     \textbf{Zeit} &  16\\
     \hline
     \textbf{Ressource} & Manherz, Klein\\
     \hline
     \textbf{Inhalt} &  Softwaredokumentation\\
     \hline
     \textbf{Abhängigkeit} & Geschriebener Code\\
\hline
\end{tabularx}

\subsection{Arbeitspaket 10 Code Review}
\begin{tabularx}{\textwidth}{|l|X|}
\hline
     \textbf{Zeit} & 40\\
     \hline
     \textbf{Ressource} & Alle\\
     \hline
     \textbf{Inhalt} &  Überprüfung des Codes anhand vorhandener Spezifikationen und Coding-Guidelines\\
     \hline
     \textbf{Abhängigkeit} & Geschriebener Code\\
\hline
\end{tabularx}

\subsection{Arbeitspaket 11 Tests}
\begin{tabularx}{\textwidth}{|l|X|}
\hline
     \textbf{Zeit} & 30\\
     \hline
     \textbf{Ressource} & Manherz, Wirth\\
     \hline
     \textbf{Inhalt} & Test der Funktionsfähigkeit von einzelnen Komponenten sowie Fehlersuche\\
     \hline
     \textbf{Abhängigkeit} & Geschriebener Code\\
\hline
\end{tabularx}

\subsection{Arbeitspaket 12 Meetings}
\begin{tabularx}{\textwidth}{|l|X|}
\hline
     \textbf{Zeit} & 128\\
     \hline
     \textbf{Ressource} & Alle\\
     \hline
     \textbf{Inhalt} & Regelmäßiges Treffen zur Besprechung über das Projekt, beinhaltet auch Beratungs- und Reviewtermine\\
     \hline
     \textbf{Abhängigkeit} & Funktionsfähiges Internet/Fortbewegungsmittel\\
\hline
\end{tabularx}

\subsection{Arbeitspaket 13 Abschlusspräsentation}
\begin{tabularx}{\textwidth}{|l|X|}
\hline
     \textbf{Zeit} &  8\\
     \hline
     \textbf{Ressource} & Alle\\
     \hline
     \textbf{Inhalt} &  Ausarbeitung und Einstudieren der Präsentation\\
     \hline
     \textbf{Abhängigkeit} & Fertiges Projekt\\
\hline
\end{tabularx}

\subsection{Arbeitspaket 14 Ansichten erstellen}
\begin{tabularx}{\textwidth}{|l|X|}
\hline
     \textbf{Zeit} & 20 \\
     \hline
     \textbf{Ressource} & AlHamed, Manherz\\
     \hline
     \textbf{Inhalt} & Asichten mit Button/Feld Dummys erstellen\\
     \hline
     \textbf{Abhängigkeit} & keine\\
\hline
\end{tabularx}

\subsection{Arbeitspaket 15 Benutzer registrieren}
\begin{tabularx}{\textwidth}{|l|X|}
\hline
     \textbf{Zeit} & 8 \\
     \hline
     \textbf{Ressource} & Manherz, Wirth\\
     \hline
     \textbf{Inhalt} & Verknüpfung von Button mit Funktionalität, Datenbankanbindung, wie UC Benutzer registrieren\\
     \hline
     \textbf{Abhängigkeit} & Datenbank einsatzfähig\\
\hline
\end{tabularx}

\subsection{Arbeitspaket 16 Benutzer anmelden}
\begin{tabularx}{\textwidth}{|l|X|}
\hline
     \textbf{Zeit} & 8\\
     \hline
     \textbf{Ressource} & Manherz, Wirth\\
     \hline
     \textbf{Inhalt} &  Verknüpfung von Button mit Funktionalität, Datenbankanbindung,\\
     \hline
     \textbf{Abhängigkeit} & Datenbank einsatzfähig\\
\hline
\end{tabularx}

\subsection{Arbeitspaket 17 Profil bearbeiten}
\begin{tabularx}{\textwidth}{|l|X|}
\hline
     \textbf{Zeit} & 8\\
     \hline
     \textbf{Ressource} & Manherz, Wirth\\
     \hline
     \textbf{Inhalt} &  Verknüpfung von Buttons/Eingabefeldern mit Funktionalität, Datenbankanbindung,\\
     \hline
     \textbf{Abhängigkeit} & Datenbank einsatzfähig\\
\hline
\end{tabularx}

\subsection{Arbeitspaket 18 Profil aufwerten}
\begin{tabularx}{\textwidth}{|l|X|}
\hline
     \textbf{Zeit} & 8 \\
     \hline
     \textbf{Ressource} & Manherz, Wirth\\
     \hline
     \textbf{Inhalt} &  Verknüpfung von Button mit Funktionalität, Datenbankanbindung,\\
     \hline
     \textbf{Abhängigkeit} & Datenbank einsatzfähig\\
\hline
\end{tabularx}

\subsection{Arbeitspaket 19 Grundumsatz berechnen}
\begin{tabularx}{\textwidth}{|l|X|}
\hline
     \textbf{Zeit} & 4 \\
     \hline
     \textbf{Ressource} & AlHamed, Klein \\
     \hline
     \textbf{Inhalt} &  Grundumsatz eines Benutzers berechnen und speichern\\
     \hline
     \textbf{Abhängigkeit} & Datenbank einsatzfähig \\
\hline
\end{tabularx}

\subsection{Arbeitspaket 20 Mahlzeit eingeben}
\begin{tabularx}{\textwidth}{|l|X|}
\hline
     \textbf{Zeit} & 12 \\
     \hline
     \textbf{Ressource} & AlHamed, Klein\\
     \hline
     \textbf{Inhalt} &  Verknüpfung von Buttons/Eingabefeldern mit Funktionalität, Datenbankanbindung\\
     \hline
     \textbf{Abhängigkeit} & Datenbank einsatzfähig\\
\hline
\end{tabularx}

\subsection{Arbeitspaket 21 Kalorien berechnen}
\begin{tabularx}{\textwidth}{|l|X|}
\hline
     \textbf{Zeit} & 8 \\
     \hline
     \textbf{Ressource} & AlHamed, Klein \\
     \hline
     \textbf{Inhalt} &  Berechnung und Speicherung von Kalorien einer vom Nutzer eingegebenen Mahlzeit\\
     \hline
     \textbf{Abhängigkeit} & Datenbank einsatzfähig\\
\hline
\end{tabularx}

\subsection{Arbeitspaket 22 UC Trainingsplan erstellen}
\begin{tabularx}{\textwidth}{|l|X|}
\hline
     \textbf{Zeit} & 20 \\
     \hline
     \textbf{Ressource} & AlHamed, Klein \\
     \hline
     \textbf{Inhalt} &  Verknüpfung von Buttons/Feldern mit Funktionalität, Datenbankanbindung\\
     \hline
     \textbf{Abhängigkeit} & Datenbank einsatzfähig\\
\hline
\end{tabularx}

\subsection{Arbeitspaket 23 Auto Trainingsplan erstellen}
\begin{tabularx}{\textwidth}{|l|X|}
\hline
     \textbf{Zeit} & 20 \\
     \hline
     \textbf{Ressource} & Manherz, Wirth\\
     \hline
     \textbf{Inhalt} &  Automatische Erstellung von Plänen mithilfe der dazugehörigen Nutzerdaten, Datenbankandbindung\\
     \hline
     \textbf{Abhängigkeit} & Datenbank einsatzfähig\\
\hline
\end{tabularx}

\subsection{Arbeitspaket 24 Trainingseinheit erstellen}
\begin{tabularx}{\textwidth}{|l|X|}
\hline
     \textbf{Zeit} & 20 \\
     \hline
     \textbf{Ressource} & Manherz, Wirth \\
     \hline
     \textbf{Inhalt} &  Verknüpfung von Buttons/Feldern mit Funktionalität, Datenbankanbindung\\
     \hline
     \textbf{Abhängigkeit} & Datenbank einsatzfähig\\
\hline
\end{tabularx}

\subsection{Arbeitspaket 25 Ernährungsplan erstellen}
\begin{tabularx}{\textwidth}{|l|X|}
\hline
     \textbf{Zeit} & 20 \\
     \hline
     \textbf{Ressource} & AlHamed, Klein\\
     \hline
     \textbf{Inhalt} &  Verknüpfung von Buttons/Feldern mit Funktionalität, Datenbankanbindung\\
     \hline
     \textbf{Abhängigkeit} & Datenbank einsatzfähig\\
\hline
\end{tabularx}

\subsection{Arbeitspaket 26 Statistik erstellen}
\begin{tabularx}{\textwidth}{|l|X|}
\hline
     \textbf{Zeit} & 20\\
     \hline
     \textbf{Ressource} & Manherz, Wirth\\
     \hline
     \textbf{Inhalt} & Erstellung einer Statistik aus den Benutzerdaten in der Datenbank, Datenbankanbindung\\
     \hline
     \textbf{Abhängigkeit} & Datenbank einsatzfähig\\
\hline
\end{tabularx}

\subsection{Arbeitspaket 27 Erinnerung senden}
\begin{tabularx}{\textwidth}{|l|X|}
\hline
     \textbf{Zeit} & 8\\
     \hline
     \textbf{Ressource} & Manherz, Wirth \\
     \hline
     \textbf{Inhalt} &  System sendet Erinnerung an bevorstehende Trainingseinheiten\\
     \hline
     \textbf{Abhängigkeit} & Datenbank einsatzfähig\\
\hline
\end{tabularx}

\subsection{Arbeitspaket 28 Termine bearbeiten}
\begin{tabularx}{\textwidth}{|l|X|}
\hline
     \textbf{Zeit} & 8\\
     \hline
     \textbf{Ressource} & Manherz, Wirth\\
     \hline
     \textbf{Inhalt} &  Verknüpfung von Buttons/Feldern mit Funktionalität, Datenbankanbindung\\
     \hline
     \textbf{Abhängigkeit} & Datenbank einsatzfähig\\
\hline
\end{tabularx}

\subsection{Arbeitspaket 29 Training durchführen}
\begin{tabularx}{\textwidth}{|l|X|}
\hline
     \textbf{Zeit} & 16\\
     \hline
     \textbf{Ressource} & AlHamed, Klein\\
     \hline
     \textbf{Inhalt} &  Verknüpfung von Buttons/Feldern mit Funktionalität, Datenbankanbindung\\
     \hline
     \textbf{Abhängigkeit} & Datenbank einsatzfähig\\
\hline
\end{tabularx}



\subsection{``Arbeitspaket'' 30 Puffer}
\begin{tabularx}{\textwidth}{|l|X|}
\hline
     \textbf{Zeit} &  100\\
     \hline
     \textbf{Ressource} & Alle\\
     \hline
     \textbf{Inhalt} &  Puffer für wahrscheinliche Änderungen im Zeitplan\\
     \hline
     \textbf{Abhängigkeit} & keine\\
\hline
\end{tabularx}

\section{Infrastruktur}
\begin{itemize}
    \item Visiual Studio Community 2022 IDE
    \item GitLab
    \item Qt Creator
    \item Overleaf
    \item Lucidchart
\end{itemize}
\section{Unterstützende Prozesse}
Als Tool zur Modellierung der Software Architektur in UML benutzen wir Lucidchart.
Zur Implementierung benutzen wir die Programmiersprache C++. Als Entwicklungsumgebung verwenden wir Microsoft Visual Studio, da diese uns den Windows Build Prozess erleichtert und Integrationen bzw. Tools für Git, Qt und SQLite zur Verfügung stellt.\\
Für das GUI verwenden wir das Qt Framework mit dem Qt Designer. SQLite wird für die permanente Datenhaltung verwendet.\\
Um sicherzuststellen, dass alle Dateien, sowie Entwicklungsstände wiederherstellbar sind, benutzen wir das von der OTH Regensburg bereitgestellte GitLab Repository. Sobald ein Arbeitspaket abgeschlossen ist wird es commited und reviewed. Fehler bzw. Bugs werden in unserem GitLab als Issues dokumentiert und frühestmöglich behoben.\\
Um die Qualität des Codes sicherzustellen richten wir uns nach dem \href{https://google.github.io/styleguide/cppguide.html}{Google C++ Style Guide}. Außerdem wird der gesamte Code kommentiert, um sicherzustellen, dass er für alle Gruppenmitglieder verständlich ist.\\
\\
Der Code für das Projekt wird in folgendem Git-Repository verwaltet:\\
\url{https://gitlab.oth-regensburg.de/IM/SWP_IM4/swp_sose22/team-6}\\\\
Die repository Struktur baut sich folgendermaßen auf.\\
Ausgehend vom branch "main" der branch "Development".\\
Auf dem branch "main" findet keinerlei Entwicklung statt. Dieser ist für releases reserviert.\\
Für einzelne Features/Arbeitspakete wird vom branch "Development" ein neuer branch abgezweigt.\\
Nach Fertigstellung wird dieser wirder zurück in den "Development" branch gemerged.\\
Tests können sowohl auf diesen sub-branches, als auch auf dem Development branch selbst durchgeführt werden.\\
Releases/Meilensteine/Prototypen werden auf den branch "main" zurück gemerged und werden entsprechend mit einem tag versehen.


%es fehlt noch die beschreibung der repository struktur!

\end{document}